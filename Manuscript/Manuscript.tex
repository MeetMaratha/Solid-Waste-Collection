\documentclass[12pt]{article}
\usepackage[a4paper,left=2cm, right=2cm, top=2cm, bottom=2cm]{geometry}
\usepackage{amssymb}
\usepackage{amsmath,mathtools}
\usepackage{relsize}
\usepackage{epsfig,graphicx}
\usepackage{color}
\usepackage{xcolor}
\usepackage{tikz}
\usepackage[authordate16,backend=biber]{biblatex-chicago}
\addbibresource{citation.bib}
\usepackage{color, colortbl}
% \usepackage{subfigure}
\usepackage{hyperref}
\usepackage{algorithm}
\usepackage{algorithmic}
% \usepackage{cite}
\usepackage{amsfonts}
\usepackage{textcomp}
\usepackage{xcolor}
\usepackage{multirow}
\usepackage{authblk}
\usepackage{subcaption}
\usepackage{nameref}
\usepackage{xstring}
\usepackage[labelfont=bf]{caption}

\begin{document}


\title{Efficient Waste Collection in Resource Constrainted Societies While Considering Dynamics}
\date{}
\author[1]{Marut Priyadarshi}
\author[1]{Meet Maratha}
\author[2]{Mohammad Anish}
\author[1*]{Vaibhav Kumar}
\affil[1]{marut19@iiserb.ac.in}
\affil[1]{meet19@iiserb.ac.in}
\affil[2]{mohammad19@iiserb.ac.in}
\affil[1*]{vaibhav@iiserb.ac.in}
\affil[1]{Department of Data Science and Engineering, IISER Bhopal}
\affil[2]{Department of Physics, IISER Bhopal}
\maketitle
\section{Abstract}
\pagebreak
\section{Introduction}

Solid Waste Management (SWM) is considered as one of the critical drivers of urban environmental management systems \parencite{hoornweg2012waste}.It involves collection and treatment of combined waste from households, agricultural, industrial, commercial activity, institutional, and miscellaneous, generated from the living community \parencite{GUPTA2015206}. India alone produces about 42.0 million tons of municipal solid waste annually, i.e., 1.15 lakh metric tons per day (TPD) \parencite{SHARMA2021293,GUPTA2015206}. These figures are bound to increase in the future, as cities are witnessing extreme demographic transfers, immigration, population growth, and consumption rate, which are the key reasons behind the increase in urban waste. This has become one of the most urgent concerns for agencies in India. To address the concerns and for improving the living standards, very recently Government of India (GoI) has launched various programs, e.g., Clean India Mission, Smart Cities, Amruth Cities, and Digital India. Waste management is one of the core infrastructure elements of these missions, which requires empirically driven conclusions to address the SWM related challenges \parencite{CHEELA2021419}. 

Solid Waste Collection (SWC) is the most integral activity of SWM \parencite*{GUPTA2015206}. However, the waste collection in developing countries like India is very unorganized, primarily due to resource constraints and poor planning of available resources \parencite{somani2021integrated}. Collection Vehicle Route Planning (CVRP) is a critical resource component of a waste collection system, whose planning is often not driven by analytics, resulting in poor collection efficiency  \parencite{akbarpour2021innovative}. Similar scenarios have been reported across cities of developing countries \parencite{anagnostopoulos2014effective, anagnostopoulos2015assessing, alwabli2020dynamic}. VRP requires modeling many dynamic components such as path planning, consideration of available resources, spatiotemporal demand patterns, and real-time dynamics of waste volume at collection points and Collection Vehicles (CV). A plethora of literature addresses subsets of these components in solution development across various cities \parencite{dugdhe2016efficient,chaudhari2018solid,badve2020garbage}. However, the literature has not reported a holistic waste collection system that considers them simultaneously for any region. Hence, the on-ground implementation of the present approaches is still minimal, leading to significant impacts on the operation costs and the environment \parencite{apaydin2007route, han2015waste}. Moreover, the components and their interrelationships are very complex for resource-constrained societies and therefore pose new challenges that require the researchers' urgent attention. While a significant number of works focus on CVRP optimization using smart bins and Internet Of Things (IoT), there is a noticeable lack of literature that integrates the different components of the SWC.  The components often lack real-time dynamic route optimization, especially in resource-constrained regions.

To address the challenges stated above, we propose a waste collection framework for cities, intending to find the optimal routes for the available CV while considering the dynamic variations in the waste information at collection points and that of the CV. We also discuss the influence of available resources in covering the collection bins, whose dynamic information drives the best path. Finding the best path is formulated as an optimization problem and solved using linear programming. This approach has an advantage over other traditional path planning approaches due to the considerations of field-based activities in the model. Further, the study focuses on resource-constrained societies resulting in a decision-making tool that can be scaled and applied universally.
Moreover, being dynamic, our model can calculate optimal routes, replicating the real-time scenarios. Hence, it can help fill the gap between theory and on-ground implementation. The paper has the following major contributions:

\begin{itemize}
\item Analysis of CV availability in study areas on waste collection, to demonstrate the total coverage.
\item Consideration of quantitatively simulated dynamic waste levels for CV and
collection bins for realistic outcomes.
\end{itemize}


\section{Background}

Much work has been done in different aspects of waste collection, such as route optimization, smart bins, segregation, landfill, and collection depot location optimization. In recent years the most common topic of work in this area has been Vehicle Routing Problem (VRP). The VRP-based studies have mostly addressed the problem of route optimization of collection vehicles for various cases such as landfill and collection sites allocation to minimize distance traveled by collection vehicles \parencite{kulcar1996optimizing,rathore2020location,barzehkar2019landfill}, collection point clustering to increase collection efficiency \parencite{al2021optimization} and calculating the optimal route with a fixed set of collection bins and landfill/depot locations \parencite{karadimas2008routing, amal2018sga, asefi2019mathematical,de2007decision, hannan2020waste,akhtar2017backtracking}. These studies have formulated the problems and solved them through various optimizers, algorithms and models such as ArcGIS Network Analyst \parencite{karadimas2008routing}, ArcView \parencite{,malakahmad2014solid}, Genetic Algorithm \parencite{amal2018sga}, Linear Programming \parencite{hannan2020waste} etc. \parencite{ grace2021smart}. The biggest drawback of these models remains their static nature; that is, the route is calculated at once, which may not reflect the onground scenario. Some studies have also modeled some dynamic aspects in the routing to study their impact on fuel consumption \parencite{hannan2020waste}. Other works focus purely on the theoretical part of the problem and test their models on simulated datasets \parencite{akhtar2017backtracking,anagnostopoulos2014effective, anagnostopoulos2015assessing,aleyadeh2018iot,akbarpour2021innovative}. Moreover, various studies have also implemented smart bins to track waste in cities such as Valorsul, Portugal \parencite{ramos2018smart}, Athens, Greece \parencite{karadimas2008routing}, St. Petersburg, Russia \parencite{anagnostopoulos2018stochastic,anagnostopoulos2015robust}, Salamanca, Spain \parencite{lozano2018smart}, Florance, Italy \parencite{baldo2021multi}, Porto Alegre, Brazil \parencite{de2007decision} and Mecca, Saudi Arabia \parencite{alwabli2020dynamic}. However, they lack the integration of real world aspects such as changes in routes based on various other dynamic factors such as real-time waste status of collection vehicles and their coordination with other vehicles. Hence, they are often not implemented in policy-making system. Furthermore, most of such works are developed for the cities of developed nations which are not resource constrained and have fewer challenges as compared to cities of developing nations.

Waste collection in resource-constrained societies comes with the additional challenge of allocating insufficient resources to meet the demand. Many studies have been implemented on the cities of developing nations such as India that focus on various aspects of waste collection. \cite{dugdhe2016efficient}, \cite{chaudhari2018solid}, and \cite{badve2020garbage} have highlighted the dynamism and lack of awareness as the major issues of dynamic waste collection in India. They implemented IoT based bins to plan the collection routes in real time. Their system didn't do an analysis of constraints related to resource availability, along with their coordination. \cite{chaudhary2019gis}, and \cite{sk2020optimal} utilized Geographic Information Systems (GIS) to calculate optimal waste collection routes in the city of Allahabad and Durgapur in India, respectively. These studies, however, didn't take any dynamic considerations.\cite{ogwueleke2009route} and \cite{malakahmad2014solid} focused on route optimization in the cities of Onitsha, Nigeria and Ipoh, Malaysia, respectively. \cite{rathore2020location} focused on optimal bin allocation to maximize the waste collection in the city of Bilaspur, India. \cite{vasagade2017dynamic}, \cite{medvedev2015waste}, and \cite{malapur2017iot} propose using IoT enabled devices and GIS to enable intelligent waste collection in smart cities. These works, however, do not consider any real time data to calculate the optimal routes.

Therefore, implementing dynamic scenarios to suit the requirements of resource constraint societies remains an open research problem. The reader can refer to recent reviews by \cite{belien2014municipal,sulemana2018optimal,abdallah2020artificial} for a detailed discussion on the works related to the optimization of waste collection and management systems.




\section{Problem Definition and Mathematical Formulation}


\subsection{Problem Definition}
CVRP in dynamic settings is implemented using Agent-Based Modeling (ABM) approach. A collection vehicle is an individual goal-driven agent that implements an optimization module to achieve its goal of routing on a good path. A path is calculated by minimizing the distance traveled and maximizing the waste collected while taking various real-time considerations. We have considered an Indian city to demonstrate its outcomes. To implement the model we have divided the study area into various regions. The region consists of smart collection bins used to collect waste and generally caters to a large area. These bins are assumed to be located on a road network made from edges and vertices. We simulated temporally varying waste values (0-100\%) for each of these vertices; we call them "smart bins". The simulation helped us overcome the requirement of physical sensor placement at these locations. We incorporated the inputs of stakeholder agency in the simulation purpose to keep the simulation values closer to the actual scenarios.\\
\\
\textbf{Input:}\\
\textit {Agent:} Waste collection vehicle.\\
\textit {Agents attributes:} Waste fill percentage, distance traveled.  \\
\textit {Environment:} Road network, bins, depot.  \\
\textit {Environment attributes:} Waste percentage of bins, route length, distance between each bin.\\
\textbf{Outcome:}\\
Best path for an agent\\

An agent is initially assigned to a specific region, and it only collects waste from bins located in the assigned region via the calculated optimal route. The "route” in our problem corresponds to path of the agent covering these bins and finishing its trip at a fixed depot. A path is made up of non-directional edges, and one or more edges make an “arc”. To make the implementation realistic we have applied limits on the total waste the agents can collect. When an agent collects the waste from a bin, the value of the CV is updated, and the numerical values of the distances are normalized to the same scale as the waste values to avoid any execution bias.\\ The model is executed after a specific time interval by considering updated bins waste values, CV waste fill status, and CV location to generate a route. The route is updated after every fixed time interval. A priority is given to a bin based on the distance to that bin from the agent's current location and the waste value of bin. This implies that a bin with a higher waste value is assigned a higher priority for the same distance value. When an agent collects the waste from a bin the waste fill percentage of the CV is updated.\\ 

The execution for any agent terminates when a run is complete or the CV is full. The "run” in our problem signifies the process of agents reaching the depot through calculated routes in various time intervals. As the agents execute in parallel, if an agent, in one-time interval fails to visit, or does not visit, a bin because of being full, other agents can cater to the bin. We also considered the case when bin waste values could update even when the CV is moving on an edge and hasn't reached the next bin. For such scenarios the model assumes that the CV has completed the current edge and considers the next bin as the starting point in the next time interval.  

\begin{figure}[H]
    \centering
    \textbf{(Figure 1 goes here)}
    \caption{General representation of the dynamism of the process}\label{fige}
\end{figure}


Figure \ref{fige} depicts the discussed problem definition. Figure 1a shows the starting state of the agent at time interval $T=0$ at depot, and the percentage fill values for various bins, and the agent respectively. Figure 1b-1d illustrate the further time intervals ($T=T_1,T_2...T_m$) and the dynamic changes in the movement of the agent, their updated fill percentages, updates in the bin fill percentage, and updated route after each time intervals i.e., at T1 and T2 the routes are different. Figure 1e depicts the last state at time interval $T=T_f$ when the agent finally returns to the depot. Figure \ref{fige2} depicts a case of multiple agents moving in parallel while working together to gather the waste optimally.  

\begin{figure}[H]
    \centering
    \textbf{(Figure 2 goes here)}
    \caption{General representation of the process for multiple agents}\label{fige2}
\end{figure}

Three different execution scenarios are demonstrated. The first scenario restricts the available resources, which in our case is the collection vehicles. The second case runs the model without any restriction on available resources. It demonstrates the impact of available resources on waste collection, bins covered, and distance travelled for each region and accumulated at the city scale. Lastly we emphasize the importance of dynamic considerations by comparing the outcome of dynamic with the static model.

\subsection {Mathematical Formulation}

The objective of the model to maximize the waste collected while minimizing the total distance traveled by the CV (see Eq (\ref{eq1})). We formulated the model as a mixed integer linear programming problem and solved it using Gurobi optimizer \parencite{gurobi} in polynomial time.

Table 1 defines all the variables used in the mathematical formulation.
\begin{center}
    \textbf{(Table 1 goes here)}\\
    \textbf{Table 1:} The description of variables
\end{center}

Weighted sum approach is one of the most widely used approaches in solving multi-criteria decision problems. In this approach objectives are aggregated prior to the optimization by assigning weights to the objective functions. We followed a similar methodology for solving the model, and the weights ($w_1$ and $w_2$ in Eq. (\ref{eq1})) were decided after a detailed sensitivity analysis and decision makers preferences concerning the objectives.

\begin{equation}\label{eq1}
    Obj(Maximize)=\sum_{i,j \in A} w_1 X_{ij} C_{ij} - w_2 Y_i f_i * BT
\end{equation}

Here, $X_{ij}$ is a binary variable whose value is 1 when an edge (i to j) is selected in a route and 0 otherwise. $C_{ij}$ represents the cost associated with the distance between i and j. The binary variable $Y_{i}$ is assigned the value 1 when a bin is served by a collection vehicle, and 0 otherwise. Once a collection vehicle serves a bin and collects the waste ($f_i$: fill ratio between 0 and 1, where 1 and 0 correspond to full and empty, respectively), the conversion factor "$BT$" scales the amount of waste in the bin to the scale of the collection vehicle's fill percentage scale. The variables $w_1$ and $w_2$ are the weights associated with distance traveled and waste collected. These weights sum up to one represented by constraint (2).

\begin{equation}\label{eq1.5}
    \sum_{n\in \{1,2\}} w_n = 1
\end{equation}


Constraint (\ref{eq2}) and constraint(\ref{eq3}) ensure that every route deduced after a time interval starts and ends at the depot. This also requires updating the starting bin denoted by variable 'st' in Eq (\ref{eq2}).
\begin{equation}\label{eq2}
    \sum_{j\in N}X_{st,j}=1 ; \forall st \in D ; j \in N
\end{equation}
\begin{equation}\label{eq3}
    \sum_{j\in N}X_{j,0}=1 ; \forall j \in N
\end{equation}
Eq \eqref{eq4} and Eq \eqref{eq5} ensure that visited bins are not covered again in the same run.
\begin{equation}\label{eq4}
    \sum_{i\in N}\sum_{j\in D } X_{ji}=1 ; \forall i,j \in A ; j\ne i
\end{equation}
\begin{equation}\label{eq5}
    \sum_{i\in N}\sum_{j\in D } X_{ij}=1 ; \forall i,j \in A ; j\ne i
\end{equation}

The constraint \eqref{eq7} ensures that the waste collected by a CV on an arc (collection of one or more edges) is the sum of the waste collected at every bin present on that arc. The value of the waste inside the CV is updated at the last bin, which is then used in calculating the updated path in the next time interval. This is achieved using a temporary variable $u_i$ is a temporary variable that resets to zero at the beginning of each time interval. It represents the percent the CV is full for each time interval. 

\begin{equation}\label{eq7}
    u_i+f_j*BT =u_j ; \forall i,j \in A 
\end{equation}

given: 
$$ i,j\ne 0$$
$$ i,j \ne st $$
$$  \forall i,j\in N$$

Constraint \eqref{eq8} guarantees that the amount of waste in a CV after reaching a bin is greater than or equal to the previous waste. This constraint helps verify that the CV is collecting the waste from the bins in its route without passing it.
The constraint \eqref{eq12} makes sure that at any point of time CV does not exceed its maximum capacity. 
\begin{equation}\label{eq8}
    u_i\ge f_i*BT
\end{equation}
\begin{equation}\label{eq12}
    u_i\le 100 - P_t ; \forall i \in N
\end{equation}

Constraint \eqref{eqY} confirms that the CV will not collect waste from a bin if collecting from that bin makes the CV exceed its maximum capacity. The constraint utilizes the variable $P_t$, which stores the cumulative values of $u_i$ over all the previous time intervals.

% Added this equation
\begin{equation}\label{eqY}
	\sum_{i\in N}Y_i f_i* BT\le100-P_t
\end{equation}




\section{Case Studies and Empirical Results}

\subsection{Data Preparation}
To prove the model's effectiveness, we empirically tested the model for the city of Chandigarh in India. The city is located in northwest India's foothills of Himalayas covering an area of approximately 149 km$^2$. It borders the states of Punjab and Haryana. The city is known for being the first planned city of India. However, some regions of the city also has various unplanned built-up patches such as Burail, Nayagaon, etc. The mixed built-up typology was suitable to test our methodology and thus, the city was found expedient to test the models. We generated the city's various waste collection points (location of smart bins). A total of 300 points were generated randomly using Geographic Information System (GIS) functionalities, with the constraint that the point should fall on the road networks. The road network was extracted from Open Street Map (OSM)\footnote{https://www.openstreetmap.org/} database. OpenStreetMap API was then used to calculate the distance between all bins and generate a distance matrix used in the optimization process. We applied K-Means clustering algorithm to cluster these points in a fixed set of clusters (see Figure \ref{figm}). In this paper we have selected total clusters to be three. The number can be updated based on user requirements. After clustering, region1, region2 and region3 had 95, 111 and 94 points, respectively. The clusters were considered as three regions for which CV had to be allocated. To replicate the actual field activities we have assumed that a CV starts and ends its route at the depot.


\begin{figure}[H]
    \centering
    \textbf{(Figure 3 goes here)}
    \caption{The selected bins and their clusters}\label{figm}
\end{figure}

\subsection{Execution case studies}
Experiments were carried out on a AMD A6-9220 processor which runs at 2.5GHz and utilizes 8 GB RAM. The model was implemented in Python 3.10.5 and solved with Gurobi optimizer version Gurobi 9.5.1. A total of four scenarios were implemented and the minimum and maximum solving time was around 67.0618 s having total variable count: 7041 and total constraint count: 7075 for the maximum case.
The suggestion of the decision makers (Municipal Corporation Chandigarh) was to provide equal preference to find minimum distance routes and maximizing the waste collection. Hence, we assigned  $w_1$ and $w_2$ to be 0.5 to execute the scenarios. This means that our optimization model gives equal importance to minimizing the distance and maximizing the waste collected. The maximum capacity of a CV was considered as 1000 Kg, and the maximum capacity of a smart bin was considered as 100 Kg.

\subsubsection*{Case 1: Restriction on Resources}
We applied restrictions on the available collection vehicles to highlight the importance of strategical usage of available resources in resource constraint societies. The CV values were varied from one to six, and the impact on total distance traveled, waste collected and bins covered was studied for each region and eventually for the city. The routes were calculated while considering the temporal dynamics of bin and waste level, CV positions in varying time steps.
\begin{figure}[H]
    \centering
    \textbf{(Figure 4 goes here)}
    \caption{Analysis of real-time restricted case}\label{figcom}
\end{figure}

A significant impact of available CV on collected waste was observed. The increase in amount of waste in CV directly corresponded to the fall in the total waste present in the bins in the regions, caused by the collection of waste from the bins by the CV in successive time intervals for the case of six CV per region (Figure \ref{figcom} (a) and (b)). The Table in Figure \ref{figcom} reflects the number of unvisited bins in the three regions for successive time intervals. As expected, we see the number of uncovered bins decrease as the six CV per region go further in their collection runs. The increase in collected waste for different numbers of CV per region varied almost linearly w.r.t the total distance traveled (Figure \ref{figcom} (c) and (d)). The increase in distance and collected waste is a direct result of having more CV running at the same time. For region 1 and region 3, a sharp drop in the waste collected was observed after five CV. This means these CV catered to almost all the smart bins for these regions. On the other hand region 2 still required more resources to cater to the demand. This is more evident by Table 2, which details the case of 6 CV per region to determine whether it is sufficient to satisfy the waste collection demand of regions and the city. It can be observed that all bins for regions 1 and 3 were covered by utilizing six CV, while the six CV covered only 85\% of the bins for region2. This is primarily because the value of waste for the bins in these regions was higher due to the region2 having more bins than region1 and region3.

\begin{center}
    \textbf{(Table 2 goes here)}\\
    \textbf{Table 2:} Observations for 6 CV per region
\end{center}

\subsubsection*{Case 2: Real-time, unrestricted}

Often the decision makers want to derive the requirement of resources that could cater to the whole demand. To achieve that, we relaxed the constraint on available resources to deduce the total resource requirement for achieving 100\% bin visits, with high waste collection. In the previous case, six CV would cater to all the bins for region 1 and region3. We extended the experiment for region 2 by increasing the available CV till we achieved 100\% bin coverage. It was observed that region 2 was fully covered by seven CV (see Table 3). Hence, given the existing bins, the city requirements can be fulfilled by 19 CV (see Table 3). Figure \ref{fig2} shows the calculated routes for each CV of regions to the depot. 

\begin{center}
    \textbf{(Table 3 goes here)}\\
    \textbf{Table 3:} Observation for unrestricted resources
\end{center}


\begin{figure}[H]
    \centering
    \textbf{(Figure 5 goes here)}
    \caption{Realtime Unrestricted}\label{fig2}
\end{figure}


\subsubsection*{Case 3: Comparison of real-time with static route calculation}

Our base case of route calculation considers real-time dynamics of waste (bin, CV) and CV's current position in real-time. However, the existing collection system of the city is static. Hence, we compared the real-time route calculation model with static by modifying the Eq \eqref{eq12} where the constraint will be less than 100, as shown in \eqref{eq12x}. In Eq \eqref{eq2} and Eq \eqref{eq3}, instead of \textit{st}, the beginning bin will always be 0
 (\ref{eq2x},\ref{eq3x}). 
\begin{equation}\label{eq12x}
    u_i\le 100
\end{equation}
\begin{equation}\label{eq2x}
    \sum_{j\in N}X_{0,j}=1 ; \forall j \in N
\end{equation}
\begin{equation}\label{eq3x}
    \sum_{j\in N}X_{j,0}=1 ; \forall j \in N
\end{equation}

The above constraints, when implemented result in a fixed optimal route that doesn't change with time. We executed the dynamic and static models for 3 CV per region, for a total of 9 CVs.



The routes (Figure \ref{fig4} and Figure \ref{fig5}) show the routes obtained for the static and dynamic case for 9 CV. The outcomes demonstrate that the consideration of dynamic variables result in different routes when compared to the static model.

\begin{figure}[H]
    \centering
    \textbf{(Figure 6 goes here)}
    \caption{Routes for 3 CV per region as calculated by static optimization}\label{fig4}
\end{figure}
\begin{figure}[H]
    \centering
    \textbf{(Figure 7 goes here)}
    \caption{Routes for 3 CV per region as calculated by real-time optimization}\label{fig5}
\end{figure}


We used the case of 3 CV per region to achieve a suitable middle ground to depict the difference in the routes created by both models, providing enough complexity to the problem for the two models to be fully utilized and compared while also maintaining the resource-constrained aspect of the problem. We observe that the weight of the waste collected is nearly the same for both cases, but the distance traveled by the CV in the real-time case is significantly lower than in the static case. This further emphasizes the efficiency of the real-time method over the static method.

We performed a detailed analysis of waste collected, distance traveled and the percentage of bins covered in the three regions. Since this is a resource-constrained region, the waste collected and bins covered is similar in the case of static and dynamic models. It is in the case of distance traveled where the dynamic model shows its advantage {Figure \ref{figcg1}(b), (e), (h)}. The reduction in distance traveled becomes even more pronounced when more CV are in use.

Since reducing the distance traveled is one of the most important goals of VRP, this makes real-time optimization the superior model in terms of both efficiency and features. In the static model, there is no accounting for new data. The initial route is the only route that is calculated. This, consequently, also makes it unreliable. On the other hand, the real-time model accounts for new data and creates new optimal routes, making it very adaptable and robust. Since real-time is just a modified and iterated version of the static model, the difference in computational power required between the two methods is same. Therefore, for real life applications, real-time dynamic model can be preferred method as it is able to deal with non-deterministic events that are synonymous with ground use and adapt to them, giving an uninterrupted and reliable service, while also providing greatly efficient results.

\begin{figure}[H]
	\centering
    \textbf{(Figure 8 goes here)}\label{ABCD}
	\caption{Detailed static vs dynamic performance analysis}\label{figcg1}
\end{figure}

\section{Discussion}

The collection of waste is an essential municipal service that involves large expenditures. Waste collection problems are, however, one of the most challenging operational problems to solve, as it involves a lot of complex dynamic activities. Without modeling these activities a solution often has limited on-ground implementation. Our objective in this paper is to address these challenges by simultaneously modeling the dynamic changes in bin and CV waste values to dynamically update the routes for maximizing the collected waste while covering less distance. Moreover, considering dynamic bin values replicate the realistic scenario. Very limited research has coupled such variations with the dynamically varying routes based on the model objective. The outcomes of various experiments prove that our methodology outperforms the existing static method of waste collection. Our model collected a similar amount of waste in a significantly less traveled distance. This can directly affect the carbon footprint and eventually develop sustainable societies.

Another important aspect of a sustainable system that this study addresses is strategic resource planning. Theoretically, enough resources can solve the problem. However, this can be a challenge for a resource-constrained society, as the resources are limited and their availability may not be sufficient for the amount of waste generated. A way to address this can be using alternative methods such as multiple collection runs per CV. However, this will have time limitations and can also lead to subpar waste collection, considering the dynamics of waste. The outcomes of the model execution show that using a reasonably less number of collection CV a large area can be catered with good efficiency.

Unlike the currently implemented systems with limited dynamic considerations, our solution:
\begin{itemize}
   
    \item is capable of calculating optimal paths based on the dynamic updates (real-time fill levels of smart bins, priority to bins based on fill values, CV position and its fill levels ) that generally happen in real-time.
    \item significantly improves the performance of the waste collection system in terms of distance traveled and waste collected.
    \item reduces the distance overheads by removing the need to visit redundant bins.
    \item being generic can be implemented in any city across the globe by updating the specific objectives and constraints. 
    
    \item addresses the challenges of decision-makers concerning a system that could be implemented in a realistic environment. Our approach of modeling the problem as a linear programming model with very few variables makes it ideal for integrating it with a real-time system.


\end{itemize}

The paper puts forward following major policy suggestions that can be implemented to to support the vision of creating smart sustainable cities:

\begin{itemize}
\item Inclusion of waste collection system in climate resilience plan: Climate resilience-based urban planning is at the centre of major decision-making systems. Waste collection involves trips of collection vehicles, which adds to carbon emissions. Strategic routing of available CV can not only benefit the economic aspects but can also help reduce carbon emissions. To achieve this, the government can include the waste collection system with the climate resilience plan of the city. The methodology proposed in this research can be an important component of such systems.
    
\item Implementation of smart bins for community or regions:  Door-to-door waste collection is not practiced in the majority of city/towns. The issue is even more challenging in dense urban areas with narrow lanes where accessibility of CV can be limited. However, smart bins for various unorganized and organized built-up regions can address the challenge. Smart bins with sensors that send fill details can help prioritize them, leading to better collection and routing strategies on similar lines to our method. The information of waste type automatically sensed using smart sensors can further benefit waste segregation which is another major challenge in waste management.\parencite{actionplan}

 \item On-board computation: Future smart cities will be developed using modern technologies as their backbone. Technology can immensely benefit waste collection by implementing numerous technologies in everyday collection practices. One of these valuable pieces of equipment is the on-board computation. The routing module proposed in the study integrated with Global Navigation Satellite Systems (GNSS) can be implemented on an onboard computer for generating routes. The driver can follow his route on the system and communicate with the office, notifying them of any important information. Benefits that increase driver efficiency are:
 \begin{itemize}
 \item Track routes in real-time
   \begin{itemize}
 \item relief driver can run a route without prior knowledge of it, which can reduce unnecessary time and cost.\item the generated trip data can further be used to update the routing model based on future requirements.
   \item brings accountability to the system as stakeholders (decision makers, citizens) can track the CV and plan accordingly. Moreover, the decision-makers can quantify the effectiveness of the collection process. 
    
 \end{itemize}
 \item Integration with billing systems with smart bins
   \begin{itemize}
 \item by integrating the billing system in the routing software with smart bins, customers can for charged for extra collection, thus not missing additional revenue. \item customers can also be charged for not segregating waste at the collection point, which can address the segregation process challenges and help bring accountability to the system.

\end{itemize}
\end{itemize}
\end{itemize}
\section{Conclusion}
Waste collection is one of the essential components of waste management process, comprising various interlinked components such as smart bins, dynamic routing, smart collection vehicles, and their coordination. The existing research is either focused on static models or lacks the integration of these components with realistic objectives. This paper, to fill the gaps implements a flexible real-time route optimization model that accepts and adapts to constantly updating data to provide optimal routes while maximizing the collected waste and minimizing the distance traveled by each CV implemented in an ABM environment. This makes the model suitable for onground implementations as it can take care of unforeseen circumstances and automatically adapt to them. The model was executed for the city of Chandigarh and it was found that the dynamic routes can reduce the distance traveled by upto 45\% for the same amount of waste collected using existing static methods. Various execution cases to support the waste collection process in resource constrained societies show the model's effectiveness in identifying the required resources to satisfy the demand in dynamic environments. 

The outcomes as a planning tool can help make decisions concerning the compromises for limited resources and their impact on waste collection and extra distance traveled to fulfill the demand. One of the study's limitations would be the non-consideration of a bin by any other CV, even if the bin were not full when visited. This can be addressed by relaxing the constraint, and its impact on outcomes can be examined. We have considered simulated smart bins for testing models, which can be replaced with IoT-enabled smart bins in real environments. Further integration of real-time data of accidents, construction work, etc., can provide more accurate routes.


\subsubsection*{Ethics declarations}
The study was approved by the Ethics Committee of the Indian Institute of Science Education and
Research Bhopal, India.
\subsubsection*{Consent to participate}
The need of Informed consent was waived by the Ethics committee of the Indian Institute of Science
Education and Research Bhopal, India.
\subsubsection*{Funding}
None.
\subsubsection*{Competing interests}
The authors declare no competing interests.
\subsubsection*{Data availability}
The datasets generated during and/or analysed during the current study are available from the corresponding
author on reasonable request.

\printbibliography

\subsection*{Figures}
Figure 1: General representation of the dynamism of the process\\
Figure 2: General representation of the process for multiple agents\\
Figure 3: The selected bins and their clusters\\
Figure 4: Analysis of real-time restricted case\\
Figure 5: Realtime Unrestricted\\
Figure 6: Routes for 3 CV per region as calculated by static optimization\\
Figure 7: Routes for 3 CV per region as calculated by real-time optimization\\
Figure 8: Detailed static vs dynamic performance analysis\\

\subsection*{Tables}
Table 1: The description of variables\\
Table 2: Observations for 6 CV per region\\
Table 3: Observation for unrestricted resources\\
\end{document}